\documentclass{article}

\begin{document}

Two dice are rolled. Let:
\[A = \text{'sum of two dice equals 3'},\]
\[B = \text{'sum of two dice equals 7'},\]
\[C = \text{'at least one of the dice shows a 1'}.\]

We are asked to find the following probabilities:

\textbf{1.} Probability of A given C (\(P(A|C)\)):
To find \(P(A|C)\), we need to determine the probability of event A occurring given that event C has occurred.
We know that event C occurs if at least one of the dice shows a 1. There are 36 equally likely outcomes when two dice are rolled, and out of these, 11 outcomes have at least one 1.
Out of these 11 outcomes, only 2 outcomes result in a sum of 3 (1+2 and 2+1).
Therefore, the probability of A given C is:
\[P(A|C) = \frac{{\text{{number of outcomes where both A and C occur}}}}{{\text{{number of outcomes where C occurs}}}} = \frac{2}{11}.\]

\textbf{2.} Probability of B given C (\(P(B|C)\)):
To find \(P(B|C)\), we need to determine the probability of event B occurring given that event C has occurred.
We know that event C occurs if at least one of the dice shows a 1. Again, out of 36 equally likely outcomes, 11 outcomes have at least one 1.
Out of these 11 outcomes, none result in a sum of 7.
Therefore, the probability of B given C is:
\[P(B|C) = \frac{{\text{{number of outcomes where both B and C occur}}}}{{\text{{number of outcomes where C occurs}}}} = 0.\]

\textbf{3.} Independence of A and C:
A and C are independent if and only if \(P(A \cap C) = P(A) \cdot P(C)\).
The probability of A is \(\frac{2}{36}\) (there are 2 outcomes resulting in a sum of 3 out of 36 total outcomes).
The probability of C is \(\frac{11}{36}\) (there are 11 outcomes with at least one 1 out of 36 total outcomes).
The probability of A and C occurring together (intersection) is \(\frac{2}{36}\).
Checking if \(P(A \cap C) = P(A) \cdot P(C)\):
\[\frac{2}{36} \neq \left(\frac{2}{36}\right) \cdot \left(\frac{11}{36}\right).\]
Since the equation does not hold, A and C are not independent.

\textbf{4.} Independence of B and C:
B and C are independent if and only if \(P(B \cap C) = P(B) \cdot P(C)\).
The probability of B is \(\frac{6}{36}\) (there are 6 outcomes resulting in a sum of 7 out of 36 total outcomes).
The probability of C is \(\frac{11}{36}\) (there are 11 outcomes with at least one 1 out of 36 total outcomes).
The probability of B and C occurring together (intersection) is 0 (no outcomes have both events occurring).
Checking if \(P(B \cap C) = P(B) \cdot P(C)\):
\[0 = \left(\frac{6}{36}\right) \cdot \left(\frac{11}{36}\right).\]
Since the equation holds, B and C are independent.

\end{document}
